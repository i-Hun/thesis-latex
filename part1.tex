\chapter{Место text mining в структуре исследовательских методов} \label{chapt1}
\section{История развития статистических методов} \label{sect1_1}
%С давних времён люди изучали вопросы сбора, измерения и анализа количественных данных, что составляет суть статистики. В Древнем Китае и Греции собирались массовые данные для того, чтобы помочь правителям в проведении фискальной или военной политики (термин <<статистика>> (от англ. state --- госудраство) был введён только в 1746 году, раньше эта область знания называлась <<госудраствоведение>>). 
%В XVI---XVII веках статистикой и присущей ей проблемой вероятностей заинтересовались математики, что привело образовании математической статистики и разработке известных нам методов статистического анализа.
\subsection{Дуальность статистики}\label{subsect1_1_1}
%Роберт Нисбет, Джон Элдер и Гари Майнер выделяют четыре поколения методов статистического анализа. Первое поколение включает себя два направления: классическую и байесову статистику. Для классического направления, в~русле которого работали Гаусс и Лаплас, было характероно рассмотрение прежде всего совместной вероятности, а для байесовой --- условной вероятности. 
Дуальность статистики берёт своё начало из философского спора Аристотеля и Платона \cite[стр. 7]{handbook_stat_dm}. Аристотель считал, что реальность может быть познана только эмпирически и что исследователь должен тщательно изучать вещественный мир вокруг себя. Он пришёл к убеждению, что можно разложить сложную систему на элементы, детально описать эти элементы, соединить их вместе и, затем, понять целое. Именно таким механистичным путём долгое время следовала наука. Однако в дальнейшем стало понятно, что не всегда целое можно представить как простую сумму частей, его составляющих. Часто, будучи соединёнными вместе, совокупность этих частей приобретает новое качество.

В отличие от своего ученика, Платон считал что свойством подлинного бытия обладают только идеи, а человек может лишь воспринимать и воплощать в вещах их смутные очертания. Для Платона идея (целое) была большим, чем сумма её материальных проявлений.

Эта дихотомия восприятия реальности проявляется во многих аспектах человеческой мысли, в том числе и в сфере статистического знания. С XVIII в. теория статистического вывода развивается в двух основных направлениях: классическом, связанном с именами Дж.~Неймана и Е.~С.~Пирсона, (и являющимся его развитием частотным) и байесовском. Сторонники классического подхода исходят из того, что истинные параметры модели не случайны, а аппроксимирующие их оценки случайны, поскольку они являются функциями наблюдений, содержащих случайный элемент. \cite[стр. 5-6]{Zellner1980} Параметры модели считаются не случайными из-за того, что классическое определение вероятности исходит из предположения равновозможности как объективного свойства изучаемых явлений, основанного на их реальной симметрии \cite[стр. 24]{Gnedenko2005}. Суждение вида <<Вероятность выпадения шестёрки при бросании игрального кубика равняется 1/6>> основывается на том, что любая из шести граней при подбрасывании на удачу не имеет реальных преимуществ перед другими, и это не подлежит формальному определению. Таким образом, вероятностью случайного события $A$ в её классическом понимании будет называться отношение числа несовместимых (не могущих произойти одновременно) и равновозможных элементарных событий $m$ к числу всех возможных элементарных событий $n$:
\begin{eqnarray}
P(A)=\frac{m}{n}
\end{eqnarray}
Однако такое определение наталкивается на некоторые непреодолимые препятствия, связанные с тем, что не все явления подчиняются принципу симметрии. Например, из соображений симметрии невозможно определить вероятность рождения ребёнка определённого пола. Для преодоления подобных трудностей был предложен статистический или частотный способ приближённой оценки неизвестной вероятности случайного события, основанный на длительном наблюдении над проявлением или непроявлением события $A$ при большом числе независимых испытаний и поиске устойчивых закономерностей числа проявлений этого события. Если в результате достаточно многочисленных наблюдений замечено, что частота события $A$ колеблется около некоторой постоянной, то мы скажем, что это событие имеет вероятность. Данные тип вероятности был выражен Р. Мизесом в следующей математической формуле:
\begin{eqnarray}
p=\lim_{x\to\infty}\frac{\mu}{n}
\end{eqnarray}
где $\mu$ --- количество успешных испытаний, $n$ --- количество всех испытаний \cite[стр. 46-47]{Gnedenko2005}.

Совершенно другим взглядом на вероятность отличался байсовский подход, названный так по имени английского математика Томаса Байеса
\clearpage