\chapter*{Введение}							% Заголовок
\addcontentsline{toc}{chapter}{Введение}	% Добавляем его в оглавление
\epigraph{В грамм добыча, в годы труды.\\
Изводишь единого слова ради\\
Тысячи тонн словесной руды.}{В.~В.~Маяковский}

В обыденном сознании социология у многих ассоциируется со статистикой. Вероятно, это произошло потому, что часто в своих исследованиях социологи, особенно отечественные, применяют достаточно тривиальные методы анализа данных из области классической математической статистики, такие как описательные статистики или анализ таблиц сопряжённости, используя стандартные статистические пакеты, например, SPSS, SAS и другие. Однако наука анализа данных за последние десятилетия ушла далеко вперёд. Лавинообразное увеличение количество информации связанное с развитием Интернета с одной стороны и появление высокопроизводительных компьютеров для её анализа с другой ознаменовали собой начало новой эпохи в дисциплине анализа данных. На основе наработок в области искусственного интеллекта, машинного обучения, статистики и проектировании баз данных в 80-х гг. XX века сформировалась новая междисциплинарная область знания --- Data Mining или интеллектуальный анализ данных. Особенность методов, объединяемых данным понятием, заключается в их способности извлекать из <<сырых>> данных ранее неизвестные нетривиальные знания. Системы Data Mining сейчас находятся на острие исследований и разработок в области анализа, моделирования и практического использования информации и знаний, создавая новую культуры анализа данных.

Сфера применения данных методов практически ничем не ограничена --- их можно применять везде, где имеются какие-либо данные \cite[стр. 81]{Duk2011}. Одной из таких сфер применения является интеллектуальный анализ данных --- прежде всего текста --- в социальных науках. Группа методов Data Mining, предназначенная для интеллектуального анализа неструктурированного текста объединяется под названием Text Mining. С помощью Text Mining можно получить результаты, недоступные классическим методам анализа данных, например, с высокой точностью спрогнозировать результаты выборов\footnote{Прогноз выборов в Венесуэле. URL: \url{http://vox-populi.ru/venezuala.phtml}} или предсказать популярность фильма до выхода в прокат на основе его обсуждения\footnote{Predicting the Future With Social Media. URL: \url{http://www.hpl.hp.com/research/scl/papers/socialmedia/socialmedia.pdf}}. 

Однако по оценке некоторых учёных многие российские социологи не знакомы с данными методами, что нельзя признать нормальным, поскольку <<отбрасывает>> отечественную социологию на 20-30 лет назад. Отсутствие соответствующей подготовки в области анализа данных приводит к поверхностному анализу эмпирических данных, в то время как важные и полезные неочевидные закономерности в данных «ускользают» от внимания исследователя \cite{Davydov_Knowledge}. Такое игнорирование современных методов анализа данных вполне может стать <<фатальной ошибкой>>\footnote{Давыдов А. А. Фатальная ошибка социологии. URL: \url{http://ecsocman.hse.ru/text/28973359/}} российской социологии. Сказанное позволяет считать, что работа, показывающая перспективы применения методов Data Mining в социологических исследованиях, является \textbf{актуальной}. 

\textbf{Проблема исследования} заключается недостаточности наработок в области применения методов Data Mining в социологии.

\textbf{Объект исследования} --- применение методов Data Mining в социологическом исследовании.

\textbf{Предмет исследования} --- возможности применения Text Mining для задач классификации и кластеризации неструктурированного текста в социологическом исследовании.
\clearpage