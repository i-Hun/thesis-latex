\chapter*{Введение}							% Заголовок
\addcontentsline{toc}{chapter}{Введение}	% Добавляем его в оглавление
\epigraph{В грамм добыча, в годы труды.\\
Изводишь единого слова ради\\
Тысячи тонн словесной руды.}{В.~В.~Маяковский}

В обыденном сознании социология у многих ассоциируется со статистикой. Вероятно, это произошло потому, что часто в своих исследованиях социологи, особенно отечественные, применяют достаточно тривиальные методы анализа данных, такие как описательные статистики или анализ таблиц сопряжённости, используя стандартные статистические пакеты, например, SPSS, SAS и другие. Однако наука анализа данных за последние десятилетия ушла далеко вперёд. Произошла целая смена парадигмы\cite{Davydov_Knowledge}





\clearpage