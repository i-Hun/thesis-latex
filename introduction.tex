\chapter*{Введение}							% Заголовок
\addcontentsline{toc}{chapter}{Введение}	% Добавляем его в оглавление

Последние несколько десятилетий наука анализа данных претерпевает самые существенные изменения. 

С одной стороны, появление глобальной сети Интернет и распространение персональных компьютеров привело к тому, что информации стало больше и производится она намного быстрее, до его возникновения. Значительная часть человеческой коммуникации переместилась в виртуальную сферу. Практически у каждой газеты или журнала имееются или электронная версия номера, или веб-сайт, где постоянно появляются новые материалы, происходит коммуникация пользователей между собой и с редакцией, проводятся голосования и прямые трансляции. Некоторые СМИ и вовсе отказываются от бумаги и полностью перебираются в электронный формат. Предоставляя более удобные средства потребления, хранения и поиска информации, чем традиционные печатные СМИ, Интернет становится новым центром притяжения как для издателей, так и для их аудитории.

С другой стороны, благодаря развитию технических средств и совершенствованию алгоритмов оперировать информацией стало проще. Обычный персональный компьютер теперь способен обрабатывать миллионы строк текста за считанные секунды.

Эти изменения открывают перед исследователями невиданные ранее перспективы. На основе наработок в области искусственного интеллекта, машинного обучения, статистики и проектировании баз данных в 80-х гг. XX века сформировалась новая междисциплинарная область знания --- Data Mining или интеллектуальный анализ данных. Особенность методов, объединяемых данным понятием, заключается в их способности извлекать из <<сырых>> данных ранее неизвестные нетривиальные знания. Системы Data Mining сейчас находятся на острие исследований и разработок в области анализа, моделирования и практического использования информации и знаний, создавая новую культуры анализа данных.

Сфера применения данных методов практически ничем не ограничена --- их можно применять везде, где имеются какие-либо данные \cite[стр. 81]{Duk2011}. Одной из таких сфер применения является интеллектуальный анализ данных --- прежде всего текста --- в социальных науках. Группа методов Data Mining, предназначенная для интеллектуального анализа неструктурированного текста объединяется под названием Text Mining.

В социологии анализ текстов обычно осуществляется следующими традиционными методами: дискурс-анализ,  контент-анализ, когнитивное картирование и т.п. Однако, как уже говорилось, виртуальное пространство является хранилищем огромного количества текстов. Поэтому обрабатывать и анализировать их обычными, привычными для социологов методами не  представляется возможным. Здесь на помощь социальному исследователю могут прийти методы text mining. С помощью Text Mining можно получить результаты, недоступные классическим методам анализа данных, например, с высокой точностью спрогнозировать результаты выборов\footnote{Прогноз выборов в Венесуэле. URL: \url{http://vox-populi.ru/venezuala.phtml}} или предсказать популярность фильма до выхода в прокат на основе его обсуждения в сети\footnote{Predicting the Future With Social Media. URL: \url{http://www.hpl.hp.com/research/scl/papers/socialmedia/socialmedia.pdf}}. 

Однако по оценке некоторых учёных, многие российские социологи не знакомы с данными методами, что нельзя признать нормальным, поскольку <<отбрасывает>> отечественную социологию на 20-30 лет назад. Отсутствие соответствующей подготовки в области анализа данных приводит к поверхностному анализу эмпирических данных, в то время как важные и полезные неочевидные закономерности в данных «ускользают» от внимания исследователя \cite{Davydov_Knowledge}. Такое игнорирование современных методов анализа данных вполне может стать <<фатальной ошибкой>>\footnote{Давыдов А. А. Фатальная ошибка социологии. URL: \url{http://ecsocman.hse.ru/text/28973359/}}  и привести к возникновению <<чёрной дыры>>\footnote{Орлов А. И. Черная дыра отечественной социологии. URL: \url{http://www.ssa-rss.ru/index.php?page_id=19&id=456}} в российской социологии. Сказанное позволяет считать, что работа, показывающая перспективы применения методов Data Mining в социологических исследованиях, является \textbf{актуальной}.

\textbf{Проблема исследования} заключается недостаточности наработок в области применения методов Data Mining в социологии.

\textbf{Объект исследования} --- применение методов Data Mining в социологическом исследовании.

\textbf{Предмет исследования} --- возможности применения Text Mining для задач классификации и кластеризации неструктурированного текста в социологическом исследовании.
\clearpage