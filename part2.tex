\chapter{Практическая часть. Исследование образа губернатора омской области в местных Интернет-СМИ} \label{chapt2}
\section{Построение модели исследования} \label{sect2_1}
В данной части работы мы разработаем и проведём исследование, на примере которого будут показаны возможности метода интеллектуального анализа текста в социологии. Цель исследования состоит в том, чтобы с помощью интеллектуального анализа текста Интернет-СМИ выявить некоторые характеристики дискурса о мэре г. Омска. Такими характеристиками являются:
\begin{enumerate}
\item Частота и время упоминания о мэре
\item Тема документа, в котором упомянут мэр 
% наиболее употребительные слова, тематическое моделирование. 
\item Попытка анализа настроений. Сравнение силы эмоций по выборке, где упомянут мэр и остальной.
% полноценный анализ тональности не будет использован так как нет готовых инструментов и он плохо работает на больших документах.
% Возможные варианты: 1) Выделить наиболее употребительные слова как тут: http://globalvoicesonline.org/2014/03/02/sentiment-analysis-of-russian-tweets-about-war-in-crimea/
% 2) Выделить темы, где эмоции проявляются наиболее сильно. Таким образом можно определить самые важные темы.
\end{enumerate}
\section{Анализ данных} \label{sect2_2}



\clearpage