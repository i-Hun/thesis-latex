\chapter{Практическая часть} \label{chapt2}
\section{Примеры использования методов text mining в социологии} \label{sect2_1}
В 2012 году было опубликовано исследование, посвящённое выявлению политических предпочтений бельгийских Интернет-СМИ в ситуации политического кризиса \cite{MediaCoverage2012}. Суть кризиса состояла в том, что на протяжении более чем полутора лет ведущие валлонские и фламандские партии не могли договориться о составе федерального правительства. Корпус документов, используемых в исследовании, составили 68 000 статей, опубликованные в онлайн версиях восьми крупнейших фламандских газет в период с начала 2011 года до завершения политического кризиса в октябре того же года. Помимо даты публикации, критерием выбора статьи для анализа служило наличие в ней ключевых слов. Такими ключевыми словами считались названия фламандских политических партий, имеющих покрайней мере одно место в парламенте, и имена их важнейших представителей.

Первичная обработка данных включала удаление дубликатов. Затем на основе тонального словаря из более чем 3000 прилагательных, которые чаще всего встречались в отзывах на товары и которые вручную были проранжированы по шкале полярности (1 -- позитивное, -1 -- негативное) и субъективности (0 -- объективное, 1 -- субъективное), в каждой статье был произведён анализ тональности упоминаний выбранных  политических партиях и политиках. Для этого подсчитывалась полярность каждого прилагательного в пределах двух предложений до и двух предложений после упоминания партии. Для уменьшения шума исключались прилагательные набравшие меньше 0,1 и больше -0,1 очка по шкале полярности. В результаты было выделено 360 613 оценок.

Следующий шаг в данном исследовании -- определение степени представленности и популярности политической партии. Степень представленности $coverage(e, s)$ политического субъекта $e$ в газете $s$ определялась как отношение количества статей газеты, где упоминалась данная партия, к количеству всех статей данной газеты $A_{s}$:
\begin{eqnarray}
coverage(e, s)=\frac{\# \{a|a \in A_{s} \wedge e \in a \}}{\# A_{s}}
\end{eqnarray}
Популярность $popularity(e)$ политического субъекта $e$ определялась через относительное количество голосов, отданных за неё в результате голосования в 2010 году $v(e)$:
\begin{eqnarray}
popularity(e)=\frac{v(e)}{\sum\limits_{e'\in\varepsilon}v(e')}
\end{eqnarray}
Популярность использовалась в качестве априорного распределения для рассчёта степени склонности газеты к освещению определённой политической партии. Данная сколнность определялась как разность между представленностью партии в газете и её реальной популярностю, определённой в резульате выборов:
\begin{eqnarray}
bias(e, s) = coverage(e, s) - popularity(e)
\end{eqnarray}
В результате данных манипуляций были выявлено, какие политические субъекты пользуются популярностью электронных СМИ в большей или меньшей степени, чем среди населения в целом.

Следующий шаг -- выявление тональности упоминания политических партий и их представителей. Для каждого субъекта было подсчитано количество положительных и отрицательных отзывов, составлен график изменения тональности во времени.

В резульате исследования при помощи методов анализа текстов были выявлены политические предпочтения главных фламандских новостных сайтов во время политического кризиса.

Более интеллектуальные методы были применены для выявления разиличий в освещении событий, приведших к восстанию 2011 года в Египте, египетискими государственными и негосударственными СМИ \cite{EgyptianUprising2012}. Материал для анализа составили более 29 000 новостных статей, вышедших в 2010--2011 годах. В методологической части работы был испольован такой метод тематического моделирования как латентное размещение Дирихле (LDA), с помощью которого можно выполнить задачу категоризации документов. Алгоритм сам определяет оптимальное количетсво категорий (тем) и распределяет документы между ними. % можно разъяснить работу алгоритма

Было показано, что правительственные СМИ при освещении таких событий акцентировали внимание на угрозе дестабилизации и терроризма и старались рассказывать проведении реформ в стране. Независимые же СМИ наоборот были нацелены на мобилизацию в целях противостоянию режиму и фактически игнорировали действия правительства. Таким образом, было доказано, что режим Хосни Мумбарака потерял контроль на медиадискурсом ещё до начала активной фазы протестов.

Существуют примеры использование методов text-mining и в отечественных исследованиях. Дальше всего в этой сфере продвинулась сотрудники НИУ-ВШЭ, в частности заведующая Лабораторией интернет-исследований Кольцова Елена Юрьевна. Исследовательский коллектив под её руководством в рамках проекта <<Разработка методологии сетевого и семантического анализа блогов для социологических задач>> поставил перед собой задачу выявления на больших массивах данных русскоязычной блогосферы тематические кластеры постов (о чем говорят?) и сообществ, основанные на комментировании (кто с кем говорит?), а также выяснения того, совпадают ли комментовые сообщества с тематическими кластерами (т.е. основана ли общность комментирования на общности темы?).

Тестовой тематикой являлась тема Ислама. Эмпирический материал исследования составили 7941 статей топовых блогеров Живого Журнала за период 21-23 и 24-26 декабря 2011 года и комментарии к ним, собранные с помощью паука краулера <<Blogminer>>. Выбор записей с таким временем написания был обусловлен тем, что именно в это время ожидалась реакция со стороны "населения" российской блогосферы на выборы в Государственную Думу, состоявшиеся 4 декабря.

Для анализа данных использовалась программа NodeXL. Сообщества выявлялись путём применения алгоритмов Вакита-Цуруми и Клозэ-Ньюмана-Мура в качестве контрольного.

После операции по выявлению сообществ, которая разделила полную сеть постов на отдельные подмножества исследователи отобрали несколько групп постов для качетсвенного анализа. Его целью было установать, связаны ли посты, входящие в одну группу по смыслу (тематически) или каким-либо другим образом (принадлежат перу одного или нескольких авторов).

По резульатам качественного изучения постов из автоматически составленных групп был сделан вывод, что гипотеза исследования не подтвердилась: не были найдены доказательства того, что комментовые сообщества интегрированые общими темами в Живом Журнале.

Несмотря на неподтверждение гипотезы исследования, участие в проекте дало исследователям богатый опыт в организации Интернет-исследований, в результате чего была написана статья <<К методологии сбора Интернет-данных для социологического анализа>> \cite{methodlogy_internet}.
% Статья Кольцовой Применение автоматических методов анализа текстов для выявления тематической структуры Российской блогосферы ещё не доступна. Появится здесь http://www.isras.ru/4M_36.html



\clearpage