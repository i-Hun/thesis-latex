\chapter{Практическая часть. Исследование образа губернатора омской области в местных Интернет-СМИ} \label{chapt2}
\section{Построение модели исследования} \label{sect2_1}
В данной части работы мы разработаем и проведём исследование, на примере которого будут показаны возможности метода интеллектуального анализа текста в социологии. Цель исследования состоит в том, чтобы с помощью интеллектуального анализа текста Интернет-СМИ выявить некоторые характеристики дискурса о мэре г. Омска. Такими характеристиками являются:
\begin{enumerate}
\item Распределение статей с упоминанием мэра во времени. Нас будет интересовать, в какие месяцы или дни недели активизируется соответствующий дискурс. В контексте этой задачи мы сравним распределение статей во времени из генеральной совокупности и распределение статей из выборочной совокупности с тем, чтобы определить, значимо ли они различаются. В зависимости от полученных результатов можно выдвинуть гипотезы, объясняющие полученное распределение.
\item Количество комментариев. Для определения заинтересованности читателей в данной теме, можно сравнить количество комментариев к принадлежащей ей статьям с количеством комментариев к статьям из другой темы.
\item Тема документа, в котором упомянут мэр. С помощью алгоритмов тематического моделирования мы определим тематический контекст статей из выборки, изучим его распределение во времени и сравним его темами в генеральной совокупности на предмет наличия общих и особенных тем. Попутно мы оценим эффективность работы алгоритма тематического моделирования. % надо сравнить темы, которые алогоритм присвоил статьям о мэре, когда они были частью генеральной совокупнстьи. если алгоритм выделил эти статьи в собоую тему, то всё норм. А ещё надо посмотреть, в какием темы раскидало статьи о мэре, при выделение тем в генеральной совокупности. 
\item Попытка анализа настроений. Нам будет произведено сравнение силы эмоций по выборке, где упомянут мэр и остальной.
% полноценный анализ тональности не будет использован так как нет готовых инструментов и он плохо работает на больших документах.
% Возможные варианты: 1) Выделить наиболее употребительные слова как тут: http://globalvoicesonline.org/2014/03/02/sentiment-analysis-of-russian-tweets-about-war-in-crimea/
% 2) Выделить темы, где эмоции проявляются наиболее сильно. Таким образом можно определить самые важные темы.
% 3) Тональный анализ можно произвести на названиях статей. Для анализа использовать sentimental или sentistrength
\end{enumerate}
\section{Анализ данных} \label{sect2_2}



\clearpage